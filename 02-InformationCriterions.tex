% % - 02-InformationCriterions

\documentclass[PredictiveAnalytics101.tex]{subfiles} 
\begin{document}
	%================================================ %
	\begin{frame}
		\frametitle{Akaike Information Criterion}
		
		\textbf{AIC}
		\begin{itemize}
			\item Akaike's information criterionis a measure of the goodness of fit of
			an estimated statistical model. 
			\item The AIC was developed by Hirotsugu Akaike under the name of ``an information criterion" in 1971. 
			\item The AIC is a \textbf{\textit{model selection}} tool i.e. a method of comparing two
			or more candidate regression models. 
			\item The AIC methodology attempts to find the model that best explains the data with a minimum of parameters. (i.e. in keeping with the law of parsimony)
		\end{itemize}
		
	\end{frame}
	%================================================ %
	\begin{frame}
		\Large
		\begin{itemize}
			\item The AIC is calculated using the "likelihood function" and the number of parameters ( Likelihood function : not on course). 
			\item The likelihood value is generally given in code output, as a complement to the AIC.
			\item Given a data set, several competing models may be ranked according to their AIC, with the one having the lowest AIC being the best. 
			\item (Although, a difference in AIC values of less than two is considered negligible).
		\end{itemize}
		
	\end{frame}
	%================================================ %
	\begin{frame}
		\frametitle{Akaike Information Criterion}
		\Large
		\begin{itemize}
			\item The Akaike information criterion is a measure of the relative goodness of fit of a statistical model. 
			\item It was developed by Hirotsugu Akaike, under the name of "an information criterion" (AIC), and was first published by Akaike in 1974.
		\end{itemize}
		
		\bigskip
		%AIC provides a means for comparison among models—a tool for model selection.
		%\bigskip
		%AIC is good for prediction.\\
		
		\[\mbox{AIC} = 2p - 2\ln(L)\]
	\end{frame}
	%================================================ %
	\begin{frame}
		\frametitle{Akaike Information Criterion}
		\Large
		\begin{itemize}
			\item $p$ is the number of free model parameters.
			\item $L$ is the value of the Likelihood function for the model in question.
			\item For AIC to be optimal, $n$ must be large compared to $p$.\\
		\end{itemize}
	\end{frame}
	%================================================ %
\begin{frame}
\frametitle{Information Criterions}


We define two types of information criterion: the Bayesian Information
Criterion (BIC) and the Akaike Information Criterion (AIC). In AIC and BIC, we choose the model that
has the minimum value of:
\[AIC = −2log(L)+2m,\]
\[BIC = −2log(L)+mlogn\]

where
\begin{itemize}
\item L is the likelihood of the data with a certain model,
\item n is the number of observations and
\item m is the number of parameters in the model.
\end{itemize}
\end{frame}
%================================================ %
\begin{frame}
\frametitle{AIC}
\Large
\begin{itemize}
\item The Akaike information criterion is a measure of the relative \textbf{goodness of fit} of a statistical model.

\item When using the AIC for selecting the parametric model class, choose
the model for which the AIC value is lowest.
\end{itemize}
\end{frame}
%================================================ %
\begin{frame}
\frametitle{Schwarz's Bayesian Information Criterion}
An alternative to the AIC is the Schwarz BIC, which additionally takes into account the sample size $n$.

\[\mbox{BIC} = p\ln{n} - 2\ln(L)\]
\end{frame}
%================================================ %
\begin{frame}

\frametitle{AIC and BIC in Two-Step Cluster Analysis}

(Removed from Last Week's Class due to Version Update)

Two-Step Cluster Analysis guides the decision of how many clusters to retain from the data by
calculating measures-of-fit such as \textbf{\textit{Akaike’s Information Criterion (AIC)}} or \textbf{\textit{Bayes Information Criterion (BIC)}}.
\end{frame}
%================================================ %
\begin{frame}
\Large
\begin{itemize}
\item 
These are relative measures of goodness-of-fit and are used to compare different
solutions with different numbers of segments.(``Relative" means that these criteria
are not scaled on a range of, for example, 0 to 1 but can generally take any value.)

\item 
\textbf{\textit{Important}}: Compared to an alternative solution with a different number of segments, smaller
values in AIC or BIC indicate an increased fit.
\end{itemize}
\end{frame}
%================================================ %
\begin{frame}
SPSS computes solutions for different segment numbers (up to the maximum number of segments specified before) and
chooses the appropriate solution by looking for the smallest value in the chosen
criterion. However, which criterion should we choose?
\begin{itemize}
\item AIC is well-known for
overestimating the correct number of segments
\item BIC has a slight tendency
to underestimate this number.
\end{itemize}

Thus, it is worthwhile comparing the clustering
outcomes of both criteria and selecting a smaller number of segments than
actually indicated by AIC. Nevertheless, when running two separate analyses,
one based on AIC and the other based on BIC, SPSS usually renders the same
results.
\end{frame}
%================================================ %
\begin{frame}
Once you make some choices or do nothing and go with the defaults, the clusters are
formed. At this point, you can consider whether the number of clusters is ``good". If
automated cluster selection is used, SPSS prints a table of statistics for different
numbers of clusters, an excerpt of which is shown in the figure below. You are interested
in finding the number of clusters at which the Schwarz BIC becomes small , but also the change in BIC between
adjacent number of clusters is small. 

The decision of how much benefit accrued by another cluster is very subjective. In addition to the BIC, a high ratio of distance of measures is desirable. In the figure below, the number of clusters with this highest ratio is three.

%\begin{figure}[h!]
%\begin{centering}
%  % Requires \usepackage{graphicx}
%  \includegraphics[width=10cm]{TwoStep1.jpg}\\
%  \caption{Schwarz Bayesian Information Criterion}
%\end{centering}
%\end{figure}
\end{frame}

\end{document}