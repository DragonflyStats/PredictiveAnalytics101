\documentclass[PredictiveAnalytics101.tex]{subfiles} 
\begin{document}
\begin{frame}
\Large
\begin{enumerate}
\item Defining the Problem
\item Processing The Data
\item Run an initial model
\item Evaluate the initial model
\item Select a final model
\item Testing the Model
\item Use the Model
\end{enumerate}
\end{frame}
%============================================================	
\begin{frame}
\frametitle{Step 1 : Defining the Problem}
\Large
% Step 1 is defining the problem. \\ 
What question(s) are you trying to answer? Once you
understand that you need to then think about what data is available to you to answer the question:

\begin{itemize}
\item Is the data directly related to the question?
\item If it is not, can you create a proxy relationship to be able to link it?
% \item Is the data you need even available within the enterprise or elsewhere?
\end{itemize}

\end{frame}
%============================================================%
\begin{frame}
\frametitle{Step 1 : Defining the Problem}
\Large
As part of step 1, you also need to specify the inputs and outputs of the model you are going to build as
these may change as you change and tweak the model. Finally, don’t forget the most commonly
forgotten piece of any new initiative. Determine, up front, how you are going to measure the results.
\end{frame}
%============================================================%
\begin{frame}
	\frametitle{Step 1 : Defining the Problem}
	\Large
\begin{itemize}
\item What measure of accuracy are you going to use? 
\item Is that level of accuracy good enough for
the business?
\item How will you benchmark the results?
\item  What criteria are you going to use to determine success or failure?
\end{itemize}
\end{frame}
%============================================================%
\begin{frame}
	\frametitle{Step 2 : Processing the Data}
	\Large
	\begin{itemize}
\item Step 2 is more rote and technical – process the data.
\item Collect the data (more is always better in this
analyst’s mind but more does not always mean easier or better results).
\item In general, more recent data is
better and the data need to be consistent. \item Don’t skimp on cleaning the data. While this may end up
taking the most time, it is critical and erroneous data will create erroneous results.
	\end{itemize}

\end{frame}
%============================================================%
\begin{frame}
	\frametitle{Step 2 : Processing the Data}
\Large
\textbf{Transforming the Data}
\begin{itemize}
\item Transforming the
data is also worth the time and effort to improve the modeling process including such things as:
\begin{itemize}
\item[$\ast$] Converting non-numerical data to numeric (or vice versa)
\item[$\ast$] Standardizing thing such as coding, definitions, costs, combining variables, etc.
\end{itemize}
%\item Predixion gives you tools, built into the software, to help you analyze, clean, and classify your data on
% the front end.
\end{itemize}

\end{frame}
%============================================================%
\begin{frame}
	\frametitle{Step 3 : Run the Initial Model}
\Large
\begin{itemize}
\item Step 3 is running the initial model. Part of step 3 is to split the dataset into a test dataset and a
validation dataset. If you really want to be able to test the accuracy of the model once it has been built,
you need to do this.
\item The software will walk you through this and will do this for you by default, holding
back 30\% of the total data for validation testing, although it does allow you to override the default
values. 
\item This is also the step whereby you will choose the method or methods by which you want to
build the model and process the data. 
\end{itemize}

\end{frame}
%============================================================%
\begin{frame}
	\frametitle{Step 3 : Run the Initial Model}
\Large
	\begin{itemize}
\item As you become more familiar with predictive modeling and with
your own data you will find that certain types of data and certain types of analyses or problems do lend
themselves more or less to certain types of modeling. 
\item But if you are just starting out, you can use the
software to guide you in the choice of a model or simply choose to run all the models against your data.
\item Once done, run the model and move on to step 4.
	\end{itemize}

\end{frame}
%============================================================%
\begin{frame}
\frametitle{Step 4 : Evaluate the Initial Model}
\Large
\begin{itemize}
\item Step 4 is to evaluate the initial results. 
\item \textit{Are the results acceptable? Are they what you were expecting
to see? Do you understand the results?} 
\item Most Importantly - Do they answer the question you are trying to answer? 
\item If the
answer is yes, then move on to the next step.
\end{itemize}
 
\end{frame}
%============================================================%
\begin{frame}
\frametitle{Step 4 : Evaluate the Initial Model}
\Large
If the answer is no, then consider the following:
\begin{itemize}
\item Try using different algorithms/models
\item Try using different data elements or repackaging what you have
\item Consider collecting more or different data
\item Consider redefining the problem, changing the question and the means to an answer as you
better understand your data and your environment
\end{itemize}
\end{frame}
%============================================================%
\begin{frame}
\frametitle{Step 4 : Evaluate the Initial Model}
\Large
\begin{itemize}
\item Part of the learning process may be to try and not “boil the ocean” with your models.
\item Think about
setting up the model to run on a number of scenarios starting from the simple and most straightforward
and then progressing to more and more complex.
\end{itemize}

\end{frame}
%============================================================%
\begin{frame}
\frametitle{Step 5} 
Step 5 is to select the final model.
\begin{itemize}
\item Don’t be afraid to try a number of different models and then when
you are satisfied with the results, choose the best one.
\item  We will talk about means of assessing the
accuracy of your model in a bit.
\item For now, choose the final model and consider whether you want to rerun
the entire dataset against the selected model and re-examine the results
\end{itemize} 
\end{frame}
%============================================================%
\begin{frame}
\frametitle{Step 6 : Testing the Model}
\begin{itemize}
\item Step 6 is to test the final model. This is another one of those things that often seems not to get done.
\item It is important to test the final model and the only way to do so is to take the selected model and run it
against a second, unrelated dataset (e.g. - the validation dataset or the portion of the dataset that was
held back for this purpose) and assess the results.
\end{itemize}

\end{frame}
%============================================================%
\begin{frame}
	\frametitle{Step 6}
	\Large
\begin{itemize}
\item Do not tweak or change the model in any way at this
point as it will invalidate any comparison to the initial results. 
\item If the results are similar and you are
satisfied with them you can move on to the final step. 
\item If you are not, then go back (to step 3) to
reassessing the model and the data, make any necessary or desired changes and try re-running the
model again.
\end{itemize}

\end{frame}
%============================================================%
\begin{frame}
\frametitle{Step 7 : Use the Model}
\Large
\begin{itemize}
\item Step 7 is to apply the model and run the prediction. 
\item There are actually two parts to this. 
\item One is done if
you want to refine the model then you can use the output from the model to determine next steps and
potential intervention or changes. 
\item Continue to test the model as much or as often as needed. 
\end{itemize}


\end{frame}
%============================================================%
\begin{frame}
	\frametitle{Step 7 : Use the Model}
	\Large
When
you are satisfied, do two things:
\begin{itemize}
\item[1] Run the necessary measures to test the final accuracy of the model (see our next discussion
points)
\item[2] Take the output from the model and turn it back into language and output for the business
that answers the initial question you set out to answer and makes it useful/usable for them 
\end{itemize}
\end{frame}
%============================================================%
\end{document}