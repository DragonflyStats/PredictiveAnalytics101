\documentclass[PredictiveAnalytics101.tex]{subfiles} 
\begin{document} 
	\begin{frame}
SUPERVISED LEARNING

Applications in which the training data comprises examples of the input vectors along with their corresponding target vectors are known as supervised learning problems.

%=========================== %
Supervised learning is when the data you feed your algorithm is "tagged" to help your logic make decisions.

Example: Bayes spam filtering, where you have to flag an item as spam to refine the results.

\end{frame}
%=================================================================== %
\begin{frame}


UNSUPERVISED LEARNING

In other pattern recognition problems, the training data consists of a set of input vectors x without any corresponding target values. The goal in such unsupervised learning problems may be to discover groups of similar examples within the data, where it is called clustering

Pattern Recognition and Machine Learning (Bishop, 2006)

Unsupervised learning are types of algorithms that try to find correlations without any external inputs other than the raw data.

Example: datamining clustering algorithms.
	\end{frame}
%=================================================================== %
	
\end{document}