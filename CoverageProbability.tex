Coverage probability (CP)
Another user friendly measure of agreement which is related to the computation of the TDI is the so called coverage probability (CP) [11,12]. The CP describes the proportion captured within a pre-specified boundary of the absolute paired-measurement differences from two devices, i.e., the value of pκ such that P(|D| < κ) = pκ. Therefore one can find pκ for a specified boundary κ using standard methods for computing probability quantities under normal assumptions [11]:

(13)
and to obtain a CP estimate, pκ can be computed by replacing μD and σD by their REML estimate counterparts derived from model (1).

As with the TDI, the CP criterion can also be translated into a hypothesis test specification. In this case the interest is to ensure that a specified boundary of the absolute paired-measurement differences captures at least a predetermined proportion, p0:


The proposed TI method for inference about the TDI can be utilized to perform inferences about the CP estimates. From the TI in (10) it follows that

(14)
Now κ is a fixed known boundary, and our interest lies in finding a lower confidence bound for the CP estimate. Thus, one can find a lower confidence bound for a non-central Student-t proportion with confidence level 1 - α by searching the non-centrality parameter, that depends on  and hence on pκ, that satisfies

(15)
and once the non-centrality parameter  is achieved, a lower bound about the proportion pκ is found using equation (5), pκ = Φ() - Φ(-2μD/σD - ).

However, the non-centrality parameter cannot be found in a closed form, so one may use again a modified version of the binary search algorithm as follows:

1. begin with the interval [low = 0; high = 1], as pκ is bounded by the interval (0,1);

2. calculate the midpoint of the interval mid = (low + high)/2 and compute the difference ;

3. if d is greater than 0 up to a tolerance bound δ (i.e., ), then recalculate the interval [low = mid + δ; high = 1]; if it is lower than 0 up to a tolerance bound δ (i.e. ), then recalculate the interval [low = 0; high = mid - δ];

4. repeat steps 2-3 until convergence, i.e. until d satisfies .
