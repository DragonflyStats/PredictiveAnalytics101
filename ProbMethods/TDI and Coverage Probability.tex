\subsection{Total Deviation Index and Coverage Probability}
%------------------------------------------------------------------------------%

%http://statistics.unl.edu/faculty/yang/agreement.pdf
%http://www.biomedcentral.com/content/pdf/1471-2288-10-31.pdf
%------------------------------------------------------------------------------%

\citet{lin2002} proposes a measure called the `Total Deviation Index'. 
This assumes that the differences of paired measurements are a random sample from a normal distribution, 
and consequently the approach is to construct a probability interval, known as a tolerance interval, 
for these differences. A tolerance interval is a statistical range within which a specified proportion 
of the population lies.
%------------------------------------------------------------------------------%
Smaller values of $q$ indicate better agreement. $P_{0}$ is specified by the practitioner.

\citet{pkcng} generalize this approach to account for situations where the distributions are not identical, which is commonly the case.
The TDI is not consistent and may not preserve its asymptotic nominal level, and that the coverage probability approach of \citet{lin2002} is overly conservative for moderate sample sizes.
This methodology proposed by \citet{pkcng} is a regression based approach that models the mean and the variance of differences as functions of observed values of the average of the paired measurements.
These methodologies have been adopted by Mayo Clinic (Research Section).

% Link: http://mayoresearch.mayo.edu/mayo/research/biostat/sasmacros.cfm

%------------------------------------------------------------------------------%
This measure was coined by Lin as the value TDI_{1-p} = \kappa that a given fraction (1-p) of the differences between two measurement methods will be in a symmetric interval [-\kappa,kappa].
This is roughly equivalently to the numerically largest of the 1-p limits of agreement.
The measure clearly has its main applicability in equivalence testing. 

Lin gives an approximate formula for the calculations.
\Theta \left( \frac{ TDI - \mu_d}{\sigma_d} \right) - \Theta \left(  \frac{ -TDI - \mu_d}{\sigma_d} \right) = 1-p

Again, the assumption of the normality of the case-wise differences is relied upon.
%------------------------------------------------------------------------------%

The approach is illustrated in a real case example where the agreement between two instruments, a handle mercury sphygmomanometer device and an OMRON 711 automatic device, is assessed in a sample of 384 subjects where measures of systolic blood pressure were taken twice by each device. A simulation study procedure is implemented to evaluate and compare the accuracy of the approach to two already established methods, showing that the TI approximation produces accurate empirical confidence levels which are reasonably close to the nominal confidence level.
