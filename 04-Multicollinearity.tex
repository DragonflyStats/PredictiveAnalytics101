\documentclass[PredictiveAnalytics101.tex]{subfiles} 
\begin{document}
\begin{frame}
\Large
\frametitle{Multicollinearity}
\begin{itemize}
\item In multiple regression, two or more predictor variables are colinear if they show strong linear relationships. This makes estimation of regression coefficients impossible. 
\item It can also produce unexpectedly large estimated standard errors for the coefficients of the X variables involved.

\end{itemize}
\end{frame}
%====================================%
\begin{frame}
\frametitle{Multicollinearity}
\Large
\begin{itemize}
\item This is why an exploratory analysis of the data should be first done to see if any collinearity among explanatory variables exists. 
\item Multicolinearity is suggested by non-significant results in individual tests on the regression coefficients for important explanatory (predictor) variables.
\item Multicolinearity may make the determination of the main predictor variable having an effect on the outcome difficult.
\end{itemize}
\end{frame}
%====================================%

\end{document}
