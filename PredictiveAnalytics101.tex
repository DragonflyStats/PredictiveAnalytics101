- Sensitivity and Specificity
- True and False Positives and Negatives
- Predictive Value and Positive Predictive Value
- Parametric vs. Non-Parametric
- Correlation Coefficient (R) and R-Squared
- Regression
- ROC curve (aka area under the curve)

Part 1 Theory 
Accuracy , Precision , Recall, Confusion matrices, ROC curve, Specificity, Sensitivity, Misclasfication cost
Variable selection, Law of Parsimony, Occam's Razor



\begin{document}
%==============================================%
\subfile{00-leadoutslides.tex}
% A Few community announcements
%  - Big Data Spain
%  - EARL
%  - PyCon Ireland
%==============================================%
\subfile{01-introduction.tex}
%  Discussion of Predictive Analytics
%  Machine Learning

%==============================================%
\subfile{01-learningtypes.tex}

% Supervised and Unsupervised Learning
% What is the difference?
%==============================================%
\subfile{01-problemtypes.tex}

% Classification and Regression
% Binary Classification Problems
% Logistic Regression

%==============================================%
\subfile{02-clustering.tex}
% Quick Discussion of Clustering Analysis
% Customer Segmentation

%==============================================%
\subfile{02-ModelAppraisal-part1.tex}

% Confusion Matrix
% Accuracy Precision Recall
% F-measures
% Cost of MisClassification
% True Positive Rate
% Specificity and Sensitivitiy


%==============================================%
\subfile{02-ModelAppraisal-part2.tex}

% Graphical Methods
% ROC Curves
% Lift
%==============================================%
\subfile{03-LawOfParsimony.tex}
\subfile{03-ImprovingModels.tex}

\end{document}
%===============================================================================%
%- http://www.avanade.com/~/media/documents/bi-white-paper-healthcare-analytics-practical-predictive-analytics-101-may-2013.pdf

A predictive model is not magic and it is not rocket science. It is actually a mathematical representation
of reality that offers a view based on science and statistics that uses data that is currently available to
predict the risk of an adverse event (or a preferred event) at some point in the future. The models, and
there are many, look for patterns in the data that may not always be evident and which seem linked to
an outcome of some sort. Depending on the data and the model it can do so in either a supervised
learning fashion, where the model describes a relationship between a set of independent attributes and
a dependent attribute or in an unsupervised learning fashion where the model, itself, will find
relationships in the data without reference to independent or dependent variables.


%=================================================================================%

Dimensionality Reduction

Feature Selection / Variable Selection

Step 1 is defining the problem. Put simply, what question(s) are you trying to answer? Once you
understand that you need to then think about what data is available to you to answer the question:
- Is the data directly related to the question?
- If it is not, can you create a proxy relationship to be able to link it?
- Is the data you need even available within the enterprise or elsewhere?
As part of step 1, you also need to specify the inputs and outputs of the model you are going to build as
these may change as you change and tweak the model. Finally, don’t forget the most commonly
forgotten piece of any new initiative. Determine, up front, how you are going to measure the results.
- What measure of accuracy are you going to use? Is that level of accuracy good enough for
the business?
- How will you benchmark the results?
- What criteria are you going to use to determine success or failure?
\end{frame}
%============================================================%
\begin{frame}
Step 2 is more rote and technical – process the data. Collect the data (more is always better in this
analyst’s mind but more does not always mean easier or better results). In general, more recent data is
better and the data need to be consistent. Don’t skimp on cleaning the data. While this may end up
taking the most time, it is critical and erroneous data will create erroneous results. Transforming the
data is also worth the time and effort to improve the modeling process including such things as:
- Converting non-numerical data to numeric (or vice versa)
- Standardizing thing such as coding, definitions, costs, combining variables, etc.
Predixion gives you tools, built into the software, to help you analyze, clean, and classify your data on
the front end.
\end{frame}
%============================================================%
\begin{frame}
Step 3 is running the initial model. Part of step 3 is to split the dataset into a test dataset and a
validation dataset. If you really want to be able to test the accuracy of the model once it has been built,
you need to do this. The software will walk you through this and will do this for you by default, holding
back 30% of the total data for validation testing, although it does allow you to override the default
values. This is also the step whereby you will choose the method or methods by which you want to
build the model and process the data. As you become more familiar with predictive modeling and with
your own data you will find that certain types of data and certain types of analyses or problems do lend
themselves more or less to certain types of modeling. But if you are just starting out, you can use the
software to guide you in the choice of a model or simply choose to run all the models against your data.
Once done, run the model and move on to step 4.
\end{frame}
%============================================================%
\begin{frame}
Step 4 is to evaluate the initial results. Are the results acceptable? Are they what you were expecting
to see? Do you understand the results? Do they answer the question you are trying to answer? If the
answer is yes, then move on to the next step. If the answer is no, then consider the following:
- Try using different algorithms/models
- Try using different data elements or repackaging what you have
- Consider collecting more or different data
- Consider redefining the problem, changing the question and the means to an answer as you
better understand your data and your environment
Part of the learning process may be to try and not “boil the ocean” with your models. Think about
setting up the model to run on a number of scenarios starting from the simple and most straightforward
and then progressing to more and more complex.
\end{frame}
%============================================================%
\begin{frame}Step 5 is to select the final model. Don’t be afraid to try a number of different models and then when
you are satisfied with the results, choose the best one. We will talk about means of assessing the
accuracy of your model in a bit. For now, choose the final model and consider whether you want to rerun
the entire dataset against the selected model and re-examine the results
\end{frame}
%============================================================%
\begin{frame}
Step 6 is to test the final model. This is another one of those things that often seems not to get done.
It is important to test the final model and the only way to do so is to take the selected model and run it
against a second, unrelated dataset (e.g. - the validation dataset or the portion of the dataset that was
held back for this purpose) and assess the results. Do not tweak or change the model in any way at this
point as it will invalidate any comparison to the initial results. If the results are similar and you are
satisfied with them you can move on to the final step. If you are not, then go back (to step 3) to
reassessing the model and the data, make any necessary or desired changes and try re-running the
model again.
\end{frame}
%============================================================%
\begin{frame}
Step 7 is to apply the model and run the prediction. There are actually two parts to this. One is done if
you want to refine the model then you can use the output from the model to determine next steps and
potential intervention or changes. Continue to test the model as much or as often as needed. But when
you are satisfied, please do two things:
1) Run the necessary measures to test the final accuracy of the model (see our next discussion
points)
2) Take the output from the model and turn it back into language and output for the business
that answers the initial question you set out to answer and makes it useful/usable for them 


The technical term for that type of analysis is multivariate analysis. Though certainly not a
totally inclusive listing, some knowledge of the following terms will help you understand the area of
predictive modeling better and make it easier for you to judge the value and the accuracy of models that
you create. Those terms are:
- Sensitivity and Specificity
- True and False Positives and Negatives
- Predictive Value and Positive Predictive Value
- Parametric vs. Non-Parametric
- Correlation Coefficient (R) and R-Squared
- Regression
- ROC curve (aka area under the curve)
